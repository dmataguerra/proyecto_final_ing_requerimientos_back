\documentclass[a4paper, 12pt]{report}
\setlength{\headheight}{15.71667pt}
\addtolength{\topmargin}{-3.71667pt}
\usepackage[spanish]{babel}
\usepackage[utf8]{inputenc}
\usepackage{graphicx}
\usepackage{hyperref}
\usepackage{fancyhdr}
\usepackage{amsmath}
\usepackage{listings}
\usepackage{geometry}

\geometry{left=2.5cm, right=2.5cm, top=2.5cm, bottom=2.5cm}
\hypersetup{
	colorlinks=true,
	linkcolor=blue,
	urlcolor=blue,
}

% Encabezado y pie de página
\pagestyle{fancy}
\fancyhf{}
\fancyhead[L]{Documentación}
\fancyhead[R]{\leftmark}
\fancyfoot[C]{\thepage}

% Datos de portada
\title{Documentación del Proyecto: Herramienta de Tareas}
\author{David Mata Guerra\\Alejandro Macías Fonseca\\Alexis Emilio Cárdenas Camacho\\Gabriel Juárez Ramírez \\ Universidad Autónoma de Querétaro}
\date{\today}

\begin{document}
	
	\maketitle
	\tableofcontents
	\newpage
	
	% Capítulo 1 - Descripción General del Proyecto
	\chapter{Descripción General del Proyecto}
	\section{Resumen}
    El propósito de este proyecto es crear una herramienta de tareas, la cual pueda crear, editar, eliminar y marcar como completadas
    las tareas que el usuario desee. Además, se podrá visualizar las tareas que se han completado y las que están pendientes.
	
	\section{Alcance y Limitaciones}
	Alcances del proyecto:
    \begin{itemize}
        \item Gestionar el proyecto mediante la herramienta JIRA utilizando la metodología Scrum.
        \item Desarrollar un prototipado con todas las funcionalidades que se dean implementar en Figma.
        \item Hacer uso un repositorio en GitHub para el control de versiones de el frontend.
        \item Hacer uso repositorio en GitHub para el control de versiones de el backend.
        \item Crear un diagrama relacional de la base de datos.
        \item Crear una base de datos con la estructura del diagrama relacional.
        \item Desarrollar la aplicación.
        \item Lanzar la aplicación en un servidor.
        \item Realizar test a la aplicación.
        \item Desglosar los pasos del proyecto en una presentación.
    \end{itemize}

    Limitaciones del proyecto:
    \begin{itemize}
        \item El desarrollo de el proyecto tiene un tiempo limitado (3 semanas y 4 días).
        \item El proyecto no busca fines lucrativos.
        \item El proyecto se encuentra limitado a la creación de tareas.
    \end{itemize}

	
	\section{Justificación}
	El proyecto surge en base a un proyecto final solicitado en la clase de Ingeniería de Requerimientos
    de la Universidad Autónoma de Querétaro. La idea de crear un gestor de tareas es con la finalidad de poner 
    a prueba los conocimientos adquiridos en la materia y en la carrera.
	
	% Capítulo 2 - Especificación de Requisitos
	\chapter{Especificación de Requerimientos}
	\section{Requerimientos Funcionales}
    \begin{itemize}
        \item Gestión de tareas:
        \begin{itemize}
            \item Crear tareas.
            \item Editar tareas.
            \item Eliminar tareas.
            \item Marcar tareas como completadas.
        \end{itemize}
        \item Visualización de tareas:
        \begin{itemize}
            \item Interfaz intuitiva y práctica de utilizar.
            \item El usuario pueda reconocer todos los cambios realizados de manera visual.
        \end{itemize}
    \end{itemize}
	
	\section{Requerimientos No Funcionales}
	\begin{itemize}
        \item Rendimiento:
        \begin{itemize}
            \item No debe de haber errores en la logica de programacion.
            \item La aplicación debe de ser rápida y eficiente.
        \end{itemize}
    \end{itemize}
	
	\section{Casos de Uso}
	Incluya diagramas de casos de uso y descripciones.
	
	% Capítulo 3 - Análisis y Diseño del Sistema
	\chapter{Análisis y Diseño del Sistema}
	\section{Diagrama de Arquitectura}
	Incluya un diagrama de arquitectura del sistema.
	
	\section{Diagramas UML}
	Diagramas de clases, de secuencia, o de actividad según se necesite.
	
	\section{Modelo de Datos}
	Describa el modelo de base de datos y los diagramas entidad-relación.
	
	\section{Especificación de Interfaz}
	Describa la interfaz de usuario y la experiencia del usuario.
	
	% Capítulo 4 - Planificación y Gestión del Proyecto
	\chapter{Planificación y Gestión del Proyecto}
	\section{Cronograma}
	Incluya un cronograma o diagrama de Gantt.
	
	\section{Recursos y Roles}
	Defina los recursos necesarios y los roles del equipo.
	
	\section{Control de Versiones}
	Describa cómo se gestionará el control de versiones.
	
	% Capítulo 5 - Implementación
	\chapter{Implementación}
	\section{Estructura del Código}
	Describa la organización del código.
	
	\section{Convenciones de Codificación}
	Incluya las convenciones de codificación que sigue el proyecto.
	
	\section{Dependencias}
	Liste las dependencias del proyecto (librerías, frameworks, etc.).
	
	% Capítulo 6 - Pruebas y Validación
	\chapter{Pruebas y Validación}
	\section{Plan de Pruebas}
	Describa las estrategias de prueba.
	
	\section{Casos de Prueba}
	Liste y describa casos de prueba específicos.
	
	\section{Registro de Errores}
	Proporcione un formato o sistema para el registro de errores.
	
	% Capítulo 7 - Manual de Usuario
	\chapter{Manual de Usuario}
	\section{Guía de Uso}
	Instrucciones para el usuario final.
	
	\section{Resolución de Problemas Comunes}
	Resuelva dudas o problemas frecuentes.
	
	% Capítulo 8 - Manual Técnico o de Mantenimiento
	\chapter{Manual Técnico o de Mantenimiento}
	\section{Instrucciones de Mantenimiento}
	Cómo mantener y actualizar el sistema.
	
	\section{Recuperación ante Fallos}
	Procedimientos para recuperación y mantenimiento.
	
	% Capítulo 9 - Documentación de Despliegue
	\chapter{Documentación de Despliegue}
	\section{Instrucciones de Instalación}
	Cómo instalar y desplegar el sistema.
	
	\section{Requisitos de Hardware y Software}
	Describa los requisitos del entorno de producción.
	
	% Capítulo 10 - Licencias y Acuerdos
	\chapter{Licencias y Acuerdos}
	Incluya licencias de software y derechos de propiedad intelectual.
	
	% Capítulo 11 - Registro de Cambios
	\chapter{Registro de Cambios (Changelog)}
	Documente los cambios y versiones del sistema.
	
\end{document}
